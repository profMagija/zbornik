\Title{Poređenje efikasnosti simuliranja epidemije na centralnoj procesorskoj jedinici i grafičkoj procesorskoj jedinici}
\TitleEng{Comparison of epidemic simulation efficiency on a central processing unit and a graphic processing unit}
\Author{Nikola Kušlaković}
\begin{Abstract}
Ispitivan je model epidemije koji koristi metod zasnovan na agentima, baziran na
SIR modelu, jednom od osnovnih epidemioloških modela. Analiziran je način na koji
simulacija ovog modela može da se modifikuje, tako da se pored izvršavanja na
procesoru, ona izvršava i na grafičkoj kartici. Model je koristio 15 parametara koji su
mogli da se menjaju, a testiran je tako da se vidi kakve sve tipove epidemije može da
simulira. Simulacije sa malim brojem agenata su se nešto brže izvršavale na procesoru,
a simulacije sa izuzetno puno agenata znatno brže su se izvršavale na grafičkoj kartici.
Pored toga, kada su simulacije koje su izvršavane na grafičkoj kartici koristile atomične
funkcije, izvršavala su se znatno brže od simulacija koje nisu koristile te funkcije.
Povećanjem broja agenata, razlika u predviđanjima simulacija sve manje zavisi od
računarske komponente na kojoj se one izvršavaju.
\end{Abstract}
\begin{AuthorEx}
Nikola Kušlaković (2002), Novi Sad, učenik 3. razreda Gimnazije „Isidora Sekulić” u Novom Sadu

MENTOR:\\
Stefan Nožinić, inžinjer za razvoj softvera, Continental Automotive d.o.o.
\end{AuthorEx}
\begin{AbstractEng}
In this paper, an epidemic simulation model using an agent based method, based
on the SIR model is examined. The way in which the algorithm of such a model can be
modified is analyzed, so that in addition to execution on the processor, the simulation
can also be executed on the graphics card. The simulation model had 15 parameters
that could be changed and it was tested to see what types of epidemics it could
simulate. The execution time of performing such a simulation on the processor and
graphics card was also examined. In addition, it was concluded that simulations
(performed exclusively on a graphics card) that used atomic functions proved to be
extremely better than those simulations that did not use those functions. Simulations
with a small number of agents (smaller than 60 thousand) were executed a little faster
on the processor, and simulations with an extremely large number of agents (more than
250 thousand) were executed much faster on the graphics card. Also, the results of the
simulations were compared, so that we can see the difference between the predictions
obtained from the simulations that were performed on two different computer
components. Increasing the agents in the simulation reduces the difference in the
prediction of the simulations. Relative deviation is less than 2\% for simulations that have
more than 50 thousand agents.
\end{AbstractEng}

\StartDoublePaper
\label{rac.kuslakovic}

\section{Uvod}

Epidemija je prirodna pojava kod koje dolazi do širenja neke bolesti ili zaraze
među ljudima ili životinjama unutar populacije. Epidemije se često događaju unutar
malih zajednica. Većina njih nisu opasne i brzo prestaju da se šire, ali se događaju i
slučajevi gde širenje epidemije nije lako zaustaviti. Širenjem na više kontinenata, država
i zajednica, epidemija dobija status pandemije, i tada se može smatrati velikom
pretnjom. Na širenje epidemije utiču prirodni faktori kao što su: vrsta zaraze ili virusa
koji se prenosi, način prenošenja virusa ili zaraze, stepen imuniteta populacije, brzina
prenošenja i stopa smrtnosti. Pored prirodnih faktora veliku ulogu igraju i ljudski,
odnosno društveni faktori. Držanje fizičke distance, blagovremeni odlazak kod lekara,
informisanost ljudi, redovna lična higijena i vakcinacija su faktori koji drastično utiču na
to koliko će se neka epidemija brzo širiti.

\AuthorExHere

Predviđanje toka epidemije predstavlja ključni elemenat za njeno zaustavljanje.
Koliko će predviđanje biti dobro, zavisi od same simulacije, odnosno od modela koji se
koristi. Dva najčešća načina za simuliranje epidemija su: metoda zasnovana na
agentima i metoda zasnovana na populaciji (Jaffry i Treur 2008). Kod metode zasnovane
na agentima, svaka osoba (agent) unutar populacije se tretira kao nezavisna jedinka,
dok kod metode zasnovane na populaciji, cela populacija je podeljena na grupe koje se
jasno izdvajaju, i prate se interakcije između tih grupa. Većina epidemioloških simulacija
se zasniva na \textit{SIR} modelu (Kermack i McKendrick 1927). Ovo je jednostavan matematički
model koji povezuje tri skupa: skup podložnih (S – susceptible), skup zaraženih (I –
infected) i skup oporavljenih (R – recovered) jedinki. Postoji i modifikovan oblik ovog
modela, koji uključuje i skup preminulih. Model funkcioniše tako što osobe kada se
zaraze, oporave, izgube imunitet ili umru, prelaze iz jednog skupa u drugi.

Pored modela i načina simuliranja epidemije, bitno je i na kojoj računarskoj
komponenti se vrši simuliranje. Centralna procesorska jedinica (CPU) i grafička
procesorska jedinica (GPU) su dva uređaja na kojima je moguće vršiti simuliranje. Ova
dva uređaja se najvećim delom razlikuju po svojoj arhitekturi, što znači da na različite
načine obrađuju podatke. Konkretno, procesor bolje obrađuje podatke redno
(sekvencijalno), dok grafička kartica bolje radi sa podacima koji mogu paralelno da se
obrađuju.

U nekim ranijm radovima (Shekh et al. 2015), koristi se metoda zasnovana na
agentima i jednostavan oblik \textit{SIR} modela (Kovac et al. 2018). U tim radovima se ne porede
rezultati predviđanja simulacija koje su izvršavale na dve različite računarske
komponente, već se samo porede brzine izvršavanja. Korišćeni su i sistemi sa više
udruženih grafičkih i procesorskih jednica (Aaby et al. 2010; Zou et al. 2013).

Način simuliranja koji je korišćen u ovom radu inspirisan je pristupom
zasnovanom na agentima, po uzoru na referentni rad (Shekh et al. 2015), gde imamo dve
vrste agenata (odrasle i decu), pri čemu je svakom agentu dodeljeno mesto stanovanja i
mesto gde odlazi na posao. Naš model predstavlja modifikaciju originalnog modela,
tako da su određeni delovi uprošćeni, neki poboljšani, a uvedene su i neke nove stavke.
Najbitnije izmene su sledeće: tok simulacije je drugačiji, imamo samo jednu vrstu
agenata (odrasle), dodata je mogućnost da agenti posle određenog vremenskog
perioda izgube imunitet i ponovo postanu podložni zarazi, osim mesta stanovanja i
mesta gde agenti odlaze na posao imamo i popularne lokacije (npr. kafiće, restorane i
klubove). Takođe, u našoj varijanti simulacije korišćen je drugačiji način generisanja
slučajnih vrednosti, i bilo je moguće menjati veliki broj parametara simulacije (uvedeno
je 15 različitih parametara), pa su na taj način mogli da se simuliraju različiti tipovi
epidemija. U okviru analize modela upoređena su vremena izvršavanja simulacije na
centralnoj procesorskoj jedinici i grafičkoj procesorskoj jedinici. Takođe, ispitana je
validnost modela kao i razlike u predviđanjima simulacija u zavisnosti od toga da li se
izvršavaju na centralnoj procesorskoj jedinici ili grafičkoj procesorskoj jedinici.

\section{Metod}

\subsection{Opis simulacije}

Simulacija je zasnovana na metodi agenata. Svakom agentu je dodeljeno mesto
gde stanuje i mesto gde odlazi na posao. Pored toga svaki agent je imao atribut koji
govori da li on želi da smanji kontakte sa drugim agentima, odnosno ovakav agent će
samo da odlazi na posao i da se vraća kući. Agenti koji ne paze na svoje zdravlje će
odlaziti na popularne lokacije, kao što su kafići, bioskopi ili klubovi.

Pre početka same simulacije potrebno je izgenerisati sve agente i dodeliti im
lokacije na kojima žive i odlaze na posao. Nakon toga, određeni procenat agenata iz
populacije biva prebačen u stanje zaraženosti. Ovi agenti će započeti širenje zaraze
kroz populaciju, pa se nazivaju inicijalni slučajevi epidemije.

Jedan dan u simulaciji se odvija tako što bi ujutru svi agenti odlazili na posao i
tamo provodili određeni vremenski period. Po završetku radnog vremena, disciplinovani
agenti se odmah vraćaju kući, dok bi agenti koji se ne čuvaju zaraze odlazili na dve
popularne lokacije i tamo provodili određeni vremenski period, i tek posle toga bi se
vraćali svojim kućama gde bi ostajali do sledećeg jutra.

Agent se može zaraziti tako što na nekoj od lokacija ostvari interakciju sa drugim
agentom koji je već zaražen. Verovatnoća da se agent zarazi je jedan od parametara
sumilacije. Bitno je napomenuti da agent može imati interakciju sa samim sobom, čime
se modeluju slučajevi kada neko na nekoj lokaciji provodi vreme sam. Takođe, agent
može više puta sa istim agentom da stupi u interakciju. Na kraju svakog dana agent koji
je zaražen ima izvesnu verovatnoću da umre. Nakon izvesnog vremena, agenti koji su
zaraženi a nisu umrli, postaju imuni, a taj imunitet je vremenski ograničen, odnosno
traje samo određeni vremenski period, tokom kojeg ne mogu da se zaraze, niti da
zaraze druge agente. Kada mu prestane imunitet, agent postaje ponovo podložan
zarazi. Na ovaj način agenti prelaze iz jednog skupa u drugi, a u slučaju da umru, više
ne učestvuju ni u kakvim interakcijama sa drugim agentima.

\begin{algorithm*}[ht]
\caption[Algoritam]{Preudokod simulacije}\label{alg:cap}
\begin{algorithmic}[1]
    \State GenerišiAgente()
    \State PostaviAgentePoKućama()
    \State InficirajAgente()
    \While{VremeSimulacije < DužinaTrajanjaSimulacije}
        \State PremestiAgenteNaLokacije()
        \While{Brojač < BrojInterakcija}
            \State OstvariInterakcije()
            \State Brojač $\gets$ Brojač + 1
        \EndWhile
        \State Brojač $\gets$ 0
        \State VremeDana $\gets$ VremeDana + 1
        \If{VremeDana = DužinaTrajanjaDana}
            \State PromeniStatusAgentima()
            \State VremeSimulacije $\gets$ VremeSimulacije + VremeDana
            \State VremeDana $\gets$ 0
        \EndIf
    \EndWhile
\end{algorithmic}
\end{algorithm*}

\subsection{Parametri simulacije}

Simulaciju je moguće podešavati kroz 15 parametara, pri čemu je važno kako će
parametri biti podešeni. Parametri su uvek bili podešeni tako da odgovaraju stvarnoj
situaciji, osim u slučaju kada je trebalo testirati simulaciju, kada su korišćene ekstremne
vrednosti parametara kako bi se utvrdila validnost simulacije. Na primer, nije se
dešavalo sledeće: da u jednoj kući stanuje više od 15 agenata, da radno vreme traje
manje od 5 sati ili da period kada je agent zaražen traje manje od 3 dana. Menjanje
jednog parametra prouzrokuje da drugi parametar mora da se menja. Tako,
povećanjem broja agenata povećava se i broj domova i broj radnih mesta. Parametri
simulacije koje je bilo moguće menjati su sledeći:

\begin{itemize}
    \item Parametri definisani celobrojnom vrednošću:
    \begin{enumerate}
        \item Broj agenata
        \item Broj kuća gde agenti žive
        \item Broj radnih mesta
        \item Broj popularnih lokacija
        \item Broj interakcija koje agent ostvari po jednom satu
    \end{enumerate}
    \item Parametri definisani realnom vrednošću:
    \begin{enumerate}
        \setcounter{enumi}{5}
        \item Verovatnoća da se agent zarazi kada dođe u kontakt sa zaraženim agentom
        \item Verovatnoća da agent na kraju svakog dana simulacije umre (ako je zaražen)
        \item Procenat ljudi u populaciji koji su zaraženi na početku simulacije (inicijalni slučajevi)
        \item Procenat ljudi u populaciji koji osim kuće odlaze samo na posao
    \end{enumerate}
    \item Parametri definisani u danima:
    \begin{enumerate}
        \setcounter{enumi}{9}
        \item Dužina trajanja zaraze kod zaraženog agenta
        \item Dužina trajanja perioda imuniteta kod agenta koji je preležao zarazu
        \item Dužina trajanja cele simulacije
    \end{enumerate}
    \item Parametri definisani u satima:
    \begin{enumerate}
        \setcounter{enumi}{12}
        \item Dužina trajanja jednog dana simulacije
        \item Dužina trajanja radnog vremena
        \item Dužina trajanja vremena koji agenti provode na popularnim lokacijama
    \end{enumerate}
\end{itemize}

\subsection{Simulacija na centralnoj procesorskoj jedinici}

Algoritam simluacije nije morao da se prilagođava da bi se simulacija izvršavala
na procesoru. Ceo program se prevodio u mašinski kod i predavao procesoru na
izvršavanje; procesor je alocirao delove radne memorije (RAM) kako bi se privremeno
čuvali podaci simulacije, a rezultati i neki bitni podaci o simulaciji beleženi su u trajnu
memoriju, u tekstualni fajl. Svaka komanda na procesoru se izvršavala redno
(sekvencijalno) i isključivo jedno procesorsko jezgro je bilo u upotrebi.

Bitno je napomenuti da je za većinu delova simulacije bilo potrebno izgenerisati
određene slučajne brojeve. Konkretno, ako uzmemo jednog agenta i želimo da mu
nađemo drugog agenta sa iste lokacije sa kojim će imati interakciju, to ćemo da uradimo
tako što ćemo nasumično izabrati nekog agenta iz skupa svih agenata sa te lokacije.
Korišćenjem slučajnih vrednosti, možemo približno da oponašamo situacije koje se
događaju u stvarnosti.

Pošto računari ne mogu da izgenerišu „prave” slučajne brojeve, koristiti se
generator pseudoslučajnih brojeva. Za realizaciju ovog projekta korišćen je Mersenne
Twister pseudogenerator (Matsumoto i Nishimura 1998), pošto je jedan od najkorišćenijih i
najpouzdanijih. Pseudogenerator se inicijalizuje na osnovu jedinstvene vrednosti koja
predstavlja seme generatora (seed). U svim simulacijama izvršenim na procesoru bilo je
korišćeno isto seme, zato što je korišćen samo jedan generator objekat.

\subsection{Simulacija na grafičkoj procesorskoj jedinici}

Arhitektura i način funkcionisanja grafičke kartice se dosta razlikuju od
arhitekture i načina funkcionisanja procesora, pa je zbog toga bilo potrebno prilično
modifikovati algoritam simulacije. Svaka grafička kartica poseduje više računskih
jedinica (jezgara), od kojih svaka jedinica može da pozove više radnih stavki odjednom
(work-item, thread). Radne stavke su organizovane u radne grupe, gde se svaka radna
grupa izvršava na jednoj računskoj jedinici, paralelno. Broj računskih jedinica, broj
radnih stavki koje mogu da se pozovu odjednom i broj radnih stavki unutar jedne radne
grupe zavise od same grafičke kartice. Svrha svake radne stavke je izvršavanje
specijalne funkcije koja se naziva jezgro (kernel). Jezgro je program koji se, kao i kôd
namenjen za izvršavanje na procesoru, prevodi u mašinski kôd, i šalje grafičkoj kartici.

Memorija grafičke kartice je takođe drugačije struktuirana u odnosu na radnu
memoriju sistema. Imamo tri tripa memorije: privatnu, lokalnu i globalnu memoriju.
Globalna memorija je najveća, i njoj mogu da pristupe sve računske jedinice, što znači
da joj može pristupiti i svaka radna stavka. Globalna memorija je zapravo radna
memorija grafičke kartice. Lokalna memorija je memorija same računske jedinice, i ona
je brža od globalne memorije, ali je zato manja po veličini. Njoj mogu samo da pristupe
radne stavke iz iste radne grupe. Privatna memorija je memorija radnih stavki. Ona je
najmanja po veličini i njoj može samo da pristupi radna stavka kojoj je ta memorija
dodeljena.

Na slici 1 dat je šematski prikaz grafičke procesorske jedinice. Konkretno, u
ovom slučaju imamo tri računske jedinice, od kojih svaka ima aktivne po dve radne
stavke, koje ujedno pripadaju istoj radnoj grupi.

\Figure{figure_1.jpg}
    {Šematski prikaz grafičke procesorske jedinice.}
    {chematic representation of a graphics processing unit.}

Prilikom adaptiranja simulacije za grafičku karticu najpre su generisani slučajni
brojevi, za šta je, takođe, korišćen pseudogenerator Mersenne Twister. Pošto radne
stavke ne smeju da koriste isti generator objekat, svaka od njih je morala da ima svoj
generator koji je drugačije inicijalizovan, što znači da je svaki generator generisao
drugačiji niz pseudoslučajnih brojeva. Na ovaj način ne dolazi do narušavanja
konzistentnosti resursa, odnosno ne može se desiti da više radnih stavki upisuje
podatke u istu memorijsku adresu (Feautrier 2011), te se na taj način ne remeti tok
simulacije. Još jedan od načina rešavanja ovog problema, koji je takođe razmatran,
jeste da se slučajni brojevi generišu unapred, a potom upišu u globalnu memoriju
grafičke kartice. Ovaj način je brzo odbačen, jer se pokazao kao izuzetno neefikasan:
zahteva jako puno memorije, usled čega ne mogu da se upišu drugi bitniji podaci
simulacije.

Sledeći korak bio je adaptiranje algoritma simulacije, da bi se određeni delovi
algoritma mogli izvršavati paralelno. Funkcije koje služe za premeštanje agenata na
različite lokacije, ostvarivanje interakcija između agenata i menjanje statusa agentima
(premeštanje agenata u skupove oporavljenih, zaraženih i podložnih) napisane su u
obliku jezgra, i na taj način su mogle da se izvršavaju paralelno. Najkompleksnija za
implementaciju bila je funkcija koja premešta agente na različite lokacije. Za to su
korišćene specijalne atomične funkcije, koje su sprečavale da više radnih stavki menja
isti podatak u memoriji. One funkcionišu tako što „zaključaju” podatak. Dok jedana
radna stavka menja podatak, druga stavka, koja želi da pristupi tom istom podatku,
mora da čeka da prva završi sa radom nad tim podatkom. Funkcija za ostvarivanje
interakcije između agenata prilagođena je tako da svakoj radnoj stavci dodeli po jednu
lokaciju koju će da obrađuje. Pošto agenti koji nisu na istim lokacijama ne mogu da
budu u kontaktu, radne stavke neće morati da pristupaju drugim lokacijama, osim onim
koje su im dodeljene. Što se tiče funkcije koja menja status agentima, ona je izmenjena
tako da svakoj radnoj stavci dodeli po jednog agenta. Na ovaj način je simulacija
paralelizovana.

Simulacija na grafičkoj kartici tekla je na sledeći način:

\begin{enumerate}
    \item Generisanje agenata i njihovo čuvanje u radnoj memoriji
    \item Inficiranje agenata (inicijalni slučajevi epidemije)
    \item Generisanje pseudo-slučajnih brojeva i njihovo čuvanje u radnoj memoriji
    \item Prevođenje jezgra
    \item Slanje svih podataka iz radne memorije sistema u globalnu memoriju grafičke kartice
    \item Simuliranje na grafičkoj kartici (izvršavanje svih jezgara)
    \item Prepisivanje rezultata i podataka o simulaciji iz globalne memorije grafičke kartice u radnu memoriju sistema
\end{enumerate}

Bitno je napomenuti da je najsporiji deo simulacija bio pripremna faza, odnosno deo gde
se generišu agenti i pseudogenerator objekti i dalje šalju u memoriju grafičke kartice.
Prepisivanje iz globalne memorije u radnu memoriju je izuzetno sporo, zato je najbolje
bilo poslati sve podatke odjednom.

\section{Rezultati i diskusija}

\subsection{Analiza modela}

Prvo je testiran sam model, kako bi se proverilo da li simulacija daje podatke koji
imaju smisla. Parametri su menjani tako da mogu da se simuliraju različiti tipovi
epidemija. Uspešno su simulirani sledeća četiri slučaja:

\begin{itemize}
    \item Zaraza koja ima visoku smrtnost. Tada epidemija posle nekoliko dana brzo
    prestaje da se širi, jer zaraženi prebrzo umiru, pa ne dolazi do širenja zaraze.
    \item Zaraza kod koje je visoka stopa prenošenja i mala stopa smrtnosti, a agenti se
    brzo oporavljaju i imaju dugačak imuni period. Tada posle određenog
    vremenskog perioda dolazi do naglog rasta zaraženih i onda do naglog pada
    broja zaraženih (pik epidemije). Epidemija nakon određenog perioda prestaje da
    se širi, i potpuno nestaje iz populacije.
    \item Zaraza koja se nikada neće istrebiti iz populacije. Ovde je takođe visoka stopa
    prenošenja zaraze i mala stopa smrtnosti, ali je zato dugačak period zaraženosti
    agenta i imunitet je kraći. Ovde takođe dolazi do naglog rasta i pada broja
    novozaraženih, ali se posle određenog perioda zaraza ne gubi iz populacije već
    je svaki dan približno konstantan broj novozaraženih.
    \item Imamo više grupa unutar populacije i te grupe su međusobno izolovane,
    odnosno agenti iz jedne grupe ne ostvaruju interakcije sa agentima iz druge
    grupe. Ako se zaraze samo agenti iz jedne grupe, druge grupe se neće zaraziti,
    jer nema mešanja grupa.
\end{itemize}

Bitno je istaći da u simulaciji nisu implementirane bolnice niti neki vidovi izolacije
koje agenti moraju da poštuju kada se otkrije da su zaraženi. Kada se agent zarazi, on
nastavlja sa svojim dnevnim obavezama, a oni agenti koji ne žele da smanje kontakte,
odlaze i na popularne lokacije, bez obzira da li su zaraženi ili ne. Ovo je jedan od
nedostataka modela. Pored toga, ne modeluju se ni ponašanja različitih starosnih

grupa. Neće svi ljudi u populaciji da se bave istim poslom i da odlaze na iste lokacije u
isto vreme. Deca će da odlaze u škole i tamo će provoditi vreme u smenama, odrasli će
da odlaze na posao u različito vreme na različite lokacije, a stariji ljudi će većinu dana
provoditi u kućama.

\subsection{Razlike u predikcijama simulacija}

Pošto se implementacija simulacije za grafičku karticu u nekim delovima dosta
razlikovala od implementacije za procesor, bilo je potrebno međusobno uporediti
rezultate dobijene simulacijama na ova dva uređaja.

Na slici 2 prikazane su relativna odstupanja simulacija u zavisnosti od broja
agenata. Relativna odstupanja su dobijena deljenjem razlike između vrednosti dobijenih
na procesoru i na grafičkoj kartici vrednošću dobijenom na procesoru. Svaka od ovih
simulacija, i na grafičkoj kartici i na procesoru, imala je iste parametre, osim broja
agenata i broja lokacija, a sve su obuhvatale period od 60 dana.

\Figure{plot_2.pdf}
    {Relativno odstupanje simulacija izvršavanim na grafičkoj kartici u odnosu na simulacije izvršavane na procesoru. Vremenski period pokriven simulacijama iznosi 60 dana.}
    {Relative deviation between simulations performed on the graphics card compared to simulations performed on the processor. The time period covered by the simulations is 60 days.}

Vidimo da su za simulacije sa 50 hiljada i više agenata razlike u predviđanjima
vrlo male, ispod 2\%. Ove razlike su posledica različitog načina generisanja slučajnih
vrednosti koje se koriste u simulaciji. Kod simulacija koje su se izvršavale na procesoru
korišćen je samo jedan pseudogenerator objekat, dok je kod simulacija na grafičkoj
kartici korišćeno više pseudogeneratora. Simulacije izvršavane na grafičkoj kartici nisu
uvek davale iste rezultate, pa je zbog toga simulacija sa potpuno istim parametrima
pokrenuta više puta, i za konačnu vrednost rezultata je uzeta srednja vrednost. Ove
varijacije u predviđanjima kod grafičke kartice posledica su korišćenja atomičnih
funkcija, zbog kojih se svaki put drugačije raspoređuju podaci između radnih stavki.
Osim toga, radna stavka neće uvek istom brzinom obrađivati iste podatke.

\subsection{Vreme izvršavanja simulacije}

Vreme na procesoru mereno je od početka do kraja simulacije, pri čemu nije
uračunato vreme potrebno da se generišu i pripreme podaci koji će se koristiti u
simulaciji. Pritom, vreme izvršavanja simulacija na grafičkoj kartici mereno je nešto
drugačije – uračunato je samo vreme potrebno da se izvrši svako jezgro, ali nije uzeto u
obzir vreme potrebno da se podaci pošalju na memoriju grafičke kartice, kao ni vreme
potrebno da se rezultati simulacije pročitaju nazad iz memorije. Osim broja agenata,
parametar simulacije koji je takođe uzet u obzir je broj interakcija koje agent ostvari po
satu (K).

\Figure{plot_3.pdf}
    {Zavisnost vremena izvršavanja simulacije od broja agenata u slučaju malog broja interakcija (K=4) za jedno procesorsko jezgro i za grafičku karticu. Vremenski period pokriven simulacijama iznosi 20 dana. }
    {Execution time of simulation with little interactions (K=4) for different number of agents on one processor core and graphics card. The time period covered by the simulations is 60 days.}

Na slici 3.A prikazano je vreme izvršavanja simulacija na grafičkoj kartici i
procesoru. Ove simulacije su, za razliku od simulacija sa slike 2, obuhvatale period od
20 dana. Vidimo da se izvršavanje simulacije do otprilike 160 hlijada agenata brže
odvija na procesoru nego na grafičkoj kartici, a da je za dosta veći broj agenata, vreme
izvršavanja na grafičkoj kartici bitno manja od vremena izvršavanja na procesoru.
Povećanjem broja agenata, vreme izvršavanja simulacije brže raste za procesor nego
za grafičku karticu.

\Figure{plot_4.pdf}
    {Zavisnost vremena izvršavanja simulacije od broja agenata u slučaju velikog broja interakcija (K=20) za jedno procesorsko jezgro i za grafičku karticu. Vremenski period pokriven simulacijama iznosi 20 dana.}
    {Execution time of simulation with many interactions (K=20) for different number of agents on one processor core and graphics card. The time period covered by the simulations is 20 days.}

Kao i na slici 3.A, na slici 3.B je prikazano vreme izvršavanja simulacija na
grafičkoj kartici i procesoru, samo što je u drugom slučaju znatno povećan broj
interakcija između agenata, tako što je broj interakcija povećan 5 puta (K=20). To znači
da su ove simulacije bile računski znatno zahtevnije od simulacija prikazanih na slici
3.A. Do otprilike 65 hiljada agenata, procesor je nešto brži od grafičke kartice, ali za
mnogo veći broj agenata, vreme izvršavanja na grafičkoj kartici je osetno manja od
brzine izvršavanja na procesoru. Razlika u vremenu izvršavanja simulacija na grafičkoj
kartici i procesoru je ovde znatno izraženija nego kod simulacija sa malim brojem
interakcija (slika 3.A).

Jedno procesorsko jezgro je mnogo brže i efikasnije od jedne računske jedinice
grafičke kartice, a još brže i efikasnije od jedne radne stavke. Kada se izvršavaju
simulacije sa malim brojem agenata, računske jedinice grafičke kartice nisu sve
podjednako iskorišćene, pa je tada izvršavanje na procesoru brže. Povećanjem broja
agenata, povećava se i upotreba svake računske jedinice, i tada su sve računske
jedinice podjednako iskorišćene. Tada paralelno izvršavanje komandi grafičke kartice
postaje znatno efikasnije od rednog izvršavanja komandi procesora. Ovim se može
objasniti zašto su se simulacije sa manjim brojem agenata izvršavale kraće na
procesoru nego na grafičkoj kartici i zašto grafik grafičke kartice posle određenog broja
agenata naglo menja tok.

Inače, moguće je optimizovati simulacije sa malim brojem agenata, tako da se
brže izvršavaju na grafičkoj procesorskoj jedinici. Jedna od mogućnosti jeste bolja
organizacija radnih stavki po radnim grupama. Ovim bi svaka računarska jedinica bila
podjednako iskorišćena, ali ova optimizacija ne bi bitno ubrzala izvršavanje simulacije.

Osim što je upoređeno vreme izvršavanja simulacija između grafičke kartice i
procesora, vreme izvršavanja je uporđene i između simulacija sa i bez atomičnih
funkcija. Ove simulacije su se isključivo izvršavale na grafičkoj kartici. Na slici 4
prikazani su rezultati ovog poređenja na logaritamskoj skali.

\Figure{plot_5.pdf}
    {Vreme izvršavanje simulacija koje su koristile i koje nisu koristile atomične funkcije.}
    {Execution time of simulations that used and did not use atomic functions.}

Vidimo da su simulacije koje nisu koristile atomične funckije izuzetno spore u odnosu na
one koje su koristile takve funkcije. Ovo se dešava jer se deo simulacije u kom se
agenti premeštaju sa jedne lokacije na drugu, kada se atomične funkcije ne koriste,
izvršava samo na jednoj radnoj stavci. Ovaj deo simulacije nije paralelizovan, i zato je
najsporiji, što prouzrukuje da cela simulacija bude izuzetno spora.

\section{Zaključak}

Iako jednostavna, razvijena verzija simulacije bazirana na SIR modelu epidemije
i agentskom pristupu, pokazala se zadovoljavajućom. Izborom odgovarajućih vrednosti
za 15 parametara modela, moguće je simulirati različite vrste i oblike epidemija. Razlike
u predikcijama, dobijene od simulacija koje su se izvršavale sa identičnim parametrima
na grafičkoj kartici i procesoru, su male kada je broj agenata veći od 50 hiljada.
Simulacije sa izuzetno mnogo agenata (preko 250 hiljada) se znatno brže izvršavaju na
grafičkoj kartici nego na procesoru. Što se tiče vremena izvršavanja simulacija sa malim
brojem agenata (manje od 60 hiljada), one će se brže izvršavati na procesoru nego na
grafičkoj kartici, ali razlika u vremenu izvršavanja nije velika. Na osnovu ovoga
zaključujemo da ako hoćemo da simuliramo epidemiju, možemo koristiti model
predstavljen u ovom radu i najbolje je da simulaciju koja ima puno agenata i puno
interakcija izvršavamo na grafičkoj kartici, i da se tada koriste atomične funkcije.

\section{Literatura}

\Source{%
    Aaby B. G., Perumalla K. S., Seal S. K. 2010.
    Efficient simulation of agent-based models on multi-GPU and multi-core clusters.
    U \textit{Proceedings of the 3rd international icst conference on simulation tools and techniques} (ur. F. Perrone i G. Stea).
    Brisel: ICST (Institute for Computer Sciences, Social-Informatics and Telecommunications Engineering), str. 1-10.
}
\Source{%
    Feautrier P. 2011.
    Bernstein's conditions.
    U \textit{Encyclopedia of Parallel Computing} (ur. A. Padua).
    Berlin: Springer, str. 130-134.
}
\Source{%
    Jaffry S. W., Treur J. 2008.
    Agent-based and popula-tion-based simulation: A comparative case study for epidemics.
    U \textit{Proceedings of the 22nd European Conference on Modelling and Simulation}.
    ECMS (The European Council for Modelling and Simulation), str. 1-10.
}
\Source{%
    Kermack W. O., McKendrick A. G. 1927.
    A contribution to the mathematical theory of epidemics.
    U \textit{Proceedings of the royal society of london}. Series A.
    \textbf{115} (772): 700-21.
}
\Source{%
    Aaby B. G., Perumalla K. S., Seal S. K. 2010.
    Efficient simulation of agent-based models on multi-GPU and multi-core clusters.
    U \textit{Proceedings of the 3rd international icst conference on simulation tools and techniques}
    (ur. F. Perrone i G. Stea).
    Brisel: ICST (Institute for Computer Sciences, Social-Informatics and Telecommunications Engineering), str. 1-10.
}
\Source{%
    Kovac T., Haber T., Van Reeth F., Hens N. 2018.
    Heterogeneous computing for epidemiological model fitting and simulation.
    \textit{BMC bioinformatics},
    \textbf{19} (1): 101-11
}
\Source{%
    Matsumoto M., Nishimura T. 1998.
    Mersenne twister: a 623-dimensionally equidistributed uniform pseudo-random number generator.
    \textit{ACM Transactions on Modeling and Computer Simulation (TOMACS)},
    \textbf{8} (1): 3-30.
}
\Source{%
    Shekh B., De Doncker E., Prieto D. 2015.
    Hybrid multi-threaded simulation of agent-based pandemic modeling using multiple GPUs.
    U \textit{2015 IEEE International Conference on Bioinformatics and Biomedicine (BIBM)}.
    IEEE, str. 1478-85.
}
\Source{%
    Zou P., Lü Y. S., Wu L. D., Chen L. L., Yao Y. P. 2013.
    Epidemic simulation of a large-scale social contact network on GPU clusters.
    \textit{Simulation},
    \textbf{89} (10): 1154-72.
}

\EndPaper
\clearpage